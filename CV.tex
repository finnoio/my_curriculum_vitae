\documentclass[11pt,a4paper]{article}

\usepackage[margin=1in]{geometry}
\usepackage{enumitem}
\usepackage{hyperref}
\usepackage{fontspec}
\usepackage{xcolor}
\usepackage{titlesec}
\usepackage{ulem}
\usepackage{setspace}

% Set default font to sans-serif
\setmainfont{Liberation Sans}[
  BoldFont={Liberation Sans Bold},
  ItalicFont={Liberation Sans Italic},
  BoldItalicFont={Liberation Sans Bold Italic}
]

% Remove page numbers
\pagenumbering{gobble}

% Define a thicker underline command
\renewcommand{\ULthickness}{2pt} % Make underline thicker

% Style for section titles (bold with underline)
\titleformat{\section}
{\Large\bfseries}
{}
{0em}
{\uline}

\titlespacing{\section}{0pt}{12pt}{8pt}

% Custom list formatting with increased spacing between jobs
\setlist[itemize]{leftmargin=*,topsep=0pt,partopsep=0pt,parsep=0pt,itemsep=4pt}

% Define time period command with white text on black background
\newcommand{\timeperiod}[1]{%
    \hfill{\small\colorbox{black}{\textcolor{white}{\textbf{#1}}}}\par%
}

% Custom spacing
\setlength{\parindent}{0pt}
\setlength{\parskip}{0pt}

% Increase spacing between job entries
\newcommand{\jobsep}{\vspace{1.5em}}

\begin{document}

% Header (right-aligned)
\begin{flushright}
{\huge\textbf{Nikolai Dementev}}\\[4pt]
{\normalsize nidementyev@gmail.com | Moscow, Russia}
\end{flushright}

\section{EDUCATION}
\textbf{BS Power Systems Engineering}\timeperiod{Sep 2011 - Jul 2016}
\textit{Moscow Power Engineering Institute, Moscow, Russia}

\section{EXPERIENCE}
\textbf{Data Engineer}\timeperiod{Apr 2023 - Present}
\textit{Pair Finance, Berlin, Germany}
\begin{itemize}
    \item Spearheaded migration from AWS EC2 to highly scalable AWS ECS/Batch infrastructure, enabling seamless handling of exponential data growth across expanding markets
    \item Developed mission-critical metrics alerting system delivering real-time business insights to stakeholders via Slack, significantly improving issue detection and resolution times
    \item Designed and implemented a centralized REST API service for event data gathering to facilitate efficient data ingestion, enabling cross-team data integration and standardization
    \item Led successful migration from MySQL data warehouse to AWS Redshift, optimizing query performance and analytical capabilities for growing data volumes
    \item Introduced and implemented dbt (data build tool) pipelines for analytics workloads, empowering data analysts with version-controlled, tested, and documented data transformations
    \item Built comprehensive testing framework using Python and Pytest for unit and integration testing, ensuring data pipeline reliability and accuracy
    \item Implemented infrastructure as code using Terraform and AWS services (Lambda, EC2, ECS, Batch, EventBridge, CloudWatch), enabling consistent and repeatable deployments
    \item Orchestrated data workflows using Dagster, improving pipeline observability and dependency management
\end{itemize}

\jobsep
\textbf{Data Engineer}\timeperiod{Nov 2020 - Apr 2023}
\textit{Delimobil, Moscow, Russia}
\begin{itemize}
    \item Engineered a comprehensive testing framework from scratch leveraging Python and Pytest, establishing robust data quality standards and maintaining a production-like testing environment
    \item Designed and implemented scalable ELT pipelines using Python, incorporating Anchor Modeling (6NF) techniques for optimized database architecture
    \item Orchestrated data integration across diverse sources including MySQL/PostgreSQL/MongoDB, Appsflyer, Mindbox, 1C, Google Sheets, Apache Kafka, and various REST APIs
    \item Developed a custom Domain-Specific Language (DSL) that streamlined pipeline development, reducing implementation time by 500\% and improving code maintainability
    \item Architected and deployed a real-time data ingestion and analysis pipeline for dynamic pricing service across international markets, utilizing Apache Kafka and SingleStore DB
    \item Implemented CI/CD integration with GitLab CI/CD, automating deployment processes and reducing TTM by 50\%
\end{itemize}

\jobsep
\textbf{QA Engineer (Automation)}\timeperiod{Feb 2019 - Nov 2020}
\textit{waves.tech (Blockchain platform), Moscow, Russia}
\begin{itemize}
    \item Automated integration testing of blockchain node application. Writing an API (REST/gRPC) for tests. Scala+Scalatest.
    \item Automated integration testing of smart contracts, written in RIDE, and the language itself. Scala+Scalatest.
    \item Developing custom transactions generator for load testing, implementation of load tests and integrating it with CI (Jenkins).
    \item Participation in development of Java/Python library for interacting with blockhain node via REST API.
\end{itemize}

\jobsep
\textbf{QA Engineer (Automation)}\timeperiod{Dec 2017 - Feb 2019}
\textit{Raiffeisen bank, Moscow, Russia}
\begin{itemize}
    \item Building a testing framework for card processing systems (REST/SOAP services, Oracle/Postgres DB, IBM MQ). Java+TestNG.
    \item Integration/unit tests writing.
    \item CI/CD integration (Atlassian Bamboo).
\end{itemize}

\jobsep
\textbf{QA Engineer (Automation)}\timeperiod{Apr 2017 - Dec 2017}
\textit{Mel.fm (Digital media platform), Moscow, Russia}
\begin{itemize}
    \item Automated integration/regression testing (Java+TestNG+Selenium).
    \item Load testing (Yandex Tank, Apache JMeter).
    \item Simple test environment management (Jenkins).
\end{itemize}

\section{SKILLS}
\begin{description}[labelwidth=1.5cm,leftmargin=!]
    \item[Languages] Python, Java, Scala
    \item[DBMS] AWS Redshift, MySQL, PostgreSQL, Clickhouse, Vertica, Oracle, SingleStore
    \item[MQ/PubSub] Apache Kafka, AWS SQS,IBM MQ
    \item[Schedulers/Orchestrators] Apache Airflow, Dagster, AWS Batch, Cron
    \item[Testing] Pytest, Scalatest, TestNG, JUnit
    \item[CI/CD tools] Github Actions, AWS CodePipeline, Jenkins, Atlassian Bamboo, Gitlab CI/CD tools
    \item[Other] Docker, AWS
\end{description}

\end{document}
