\documentclass[11pt,a4paper]{article}

\usepackage[margin=1in]{geometry}
\usepackage{enumitem}
\usepackage{hyperref}
\usepackage{fontspec}
\usepackage{xcolor}
\usepackage{titlesec}
\usepackage{ulem}
\usepackage{setspace}

% Set default font to sans-serif
\setmainfont{Liberation Sans}[
  BoldFont={Liberation Sans Bold},
  ItalicFont={Liberation Sans Italic},
  BoldItalicFont={Liberation Sans Bold Italic}
]

% Remove page numbers
\pagenumbering{gobble}

% Style for section titles (bold with underline)
\titleformat{\section}
{\Large\bfseries}
{}
{0em}
{\uline}

\titlespacing{\section}{0pt}{12pt}{8pt}

% Custom list formatting with increased spacing between jobs
\setlist[itemize]{leftmargin=*,topsep=0pt,partopsep=0pt,parsep=0pt,itemsep=4pt}

% Define time period command with white text on black background
\newcommand{\timeperiod}[1]{%
    \hfill{\small\colorbox{black}{\textcolor{white}{\textbf{#1}}}}\par%
}

% Custom spacing
\setlength{\parindent}{0pt}
\setlength{\parskip}{0pt}

% Increase spacing between job entries
\newcommand{\jobsep}{\vspace{1.5em}}

\begin{document}

% Header (left-aligned)
{\huge\textbf{Nikolay Dementyev}}\\[4pt]
{\normalsize nidementyev@gmail.com | Moscow, Russia}

\section{EDUCATION}
\textbf{BS Thermal physics}\timeperiod{Sep 2011 - Jul 2016}
\textit{Moscow Power Engineering Institute, Moscow, Russia}

\section{EXPERIENCE}
\textbf{Data Engineer}\timeperiod{Nov 2020 - Present}
\textit{Delimobil (Russia's leading carsharing service), Moscow, Russia}
\begin{itemize}
    \item With my previous QA experience I started with data quality activities, as I developed testing framework from scratch and ran prod-like testing environment. Python+ Pytest.
    \item Implementing ELT pipelines (Python) using Anchor Modeling(6NF) as a database modeling technique.
    \item Working with many various data sources, such as MySQL/PostgreSQL/Mongo DBs, Appsflyer, Mindbox, 1C, Google Sheets, Apache Kafka, different REST API's.
    \item Developed an DSL for writing new pipelines.
    \item Integrated real-time data ingestion/analysing pipeline for dynamic pricing service on our international branch (Apache Kafka, SingleStore DB).
    \item Initiated and implemented the Airflow upgrade from version 1.x.x to docker-based 2.x.x
    \item CI/CD integration (Gitlab CI/CD)
\end{itemize}

\jobsep
\textbf{QA Engineer (Automation)}\timeperiod{Feb 2019 - Nov 2020}
\textit{waves.tech (Blockchain platform), Moscow, Russia}
\begin{itemize}
    \item Automated integration testing of blockchain node application. Writing an API (REST/gRPC) for tests. Scala+Scalatest.
    \item Automated integration testing of smart contracts, written in RIDE, and the language itself. Scala+Scalatest.
    \item Developing custom transactions generator for load testing, implementation of load tests and integrating it with CI (Jenkins).
    \item Participation in development of Java/Python library for interacting with blockhain node via REST API.
\end{itemize}

\jobsep
\textbf{QA Engineer (Automation)}\timeperiod{Dec 2017 - Feb 2019}
\textit{Raiffeisen bank, Moscow, Russia}
\begin{itemize}
    \item Building a testing framework for card processing systems (REST/SOAP services, Oracle/Postgres DB, IBM MQ). Java+TestNG.
    \item Integration/unit tests writing.
    \item CI/CD integration (Atlassian Bamboo).
\end{itemize}

\jobsep
\textbf{QA Engineer (Automation)}\timeperiod{Apr 2017 - Dec 2017}
\textit{Mel.fm (Digital media platform), Moscow, Russia}
\begin{itemize}
    \item Automated integration/regression testing (Java+TestNG+Selenium).
    \item Load testing (Yandex Tank, Apache JMeter).
    \item Simple test environment management (Jenkins).
\end{itemize}

\section{SKILLS}
\begin{description}[labelwidth=1.5cm,leftmargin=!]
    \item[Languages] Java, Scala, Python
    \item[DBs] Oracle, MySQL, PostgreSQL, MemSQL (SingleStore), MongoDB, Vertica
    \item[MQs] IBM MQ, Apache Kafka
    \item[Schedulers] Airflow, Cron
    \item[Testing] Pytest, Scalatest, TestNG, JUnit
    \item[CI/CD tools] Jenkins, Atlassian Bamboo, Gitlab CI/CD tools
    \item[Other] Docker, AWS
\end{description}

\end{document}
